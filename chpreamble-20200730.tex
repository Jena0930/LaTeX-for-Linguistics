%% layout ------------------------------------------------------

\usepackage{fullpage}
\usepackage{geometry}  % See geometry.pdf to learn the layout options. There are lots.
%\geometry{landscape} % Activate for for rotated page geometry
\setlength{\parskip}{0.6ex}
\usepackage{fancyhdr}
%\pagestyle{fancy}
\usepackage{authblk}


%% input ---------------------------------------------------------
%%language and fonts
\usepackage[utf8]{inputenc}
%This is the encoding for the document. It can be omitted or changed to another encoding but utf-8 is recommended. Unless you specifically need another encoding, or if you are unsure about it, add this line to the preamble.
\usepackage[no-math]{fontspec} %--> doesn't like tipa
%[no-math] is used for easy use of superscript and subscript
\usepackage{microtype}
%for better text justification


\usepackage[british]{babel}
%\usepackage{xeCJK} %compiler needs to be changed to Xe(La)Tex
%\setCJKmainfont{KaiTi} %Simsun 宋体
\usepackage{xpinyin}
%\xpinyinsetup{pysep={}}

\AtBeginDocument{
\newcommand{\p}[1]{\pinyin{#1}}
\newcommand{\ip}[1]{\pinyin[format={\it}]{#1}}
\newcommand{\bp}[1]{\pinyin[format={\bf}]{#1}}
}

\usepackage[normalem]{ulem} %for underline permitting line break
% \sout{} for striked out; \xout{} for crossed out; 
% \uline{} for underline; \uuline{} for double underline; \uwave{} for waved line; \dashuline{} for dashed line; \dotuline{} for dotted line
\usepackage{multicol}
% see documentation for adjusting the space between columns
\usepackage{multirow}

%%color, graphic
\usepackage[x11names]{xcolor}
\usepackage{graphicx}
% \graphicspath{...} can direct Latex to the folder of images. This line should be put under \usepackage{graphicx} in the preamble field. \includegraphics{...} is the real command used in the document.
\usepackage{url} %use to typeset URLs properly

\usepackage{placeins}
% in combination with the command \FloatBarrier to prevent floating text on top of a table/tables.

\usepackage{hyperref}
\hypersetup{
	colorlinks=true,
	linkcolor=blue,
	filecolor=magenta, 
	urlcolor=blue,
	citecolor=black,
}

\usepackage{verbatim}
% the following command enables indentation of texts in verbatim environment.
\newlength\myverbindent 
\setlength\myverbindent{1in} % change this to change indentation
\makeatletter
\def\verbatim@processline{%
	\hspace{\myverbindent}\the\verbatim@line\par}
\makeatother
%use the following two packages for line-break in verbatim environment
\usepackage{fancyvrb}
\usepackage{fvextra}
% Don't forget to add [breaklines=true] after \begin{Verbatim}. Note that v must be changed into a capital letter.
\makeatletter
\fvset{breaklines=true,tabsize=4}

%%  linguistic packages -------------------------------------------
%\usepackage{ling-macros}
%% tree ---
\usepackage[linguistics]{forest}
\AtBeginDocument{
	\forestset{
		nice nodes/.style={
			for tree={
				inner sep=1pt, s sep=12pt,
				fit=band,  %align the nodes (more or less)
			},
		},
		default preamble=nice nodes,
	}
}

% mvt drawing with forest
\def\centerarc[#1](#2)(#3:#4:#5)% Syntax: [draw options] (center) (initial angle:final angle:radius)
{ \draw[#1] ($(#2)+({#5*cos(#3)},{#5*sin(#3)})$) arc (#3:#4:#5); }
\newcommand\phase[2][]{\centerarc[#1]($(#2.center)-(90:.75)$)(35:160:1.0)}


%% glossing ---
\usepackage{gb4e} %for examples; if fed up with gb4e, try XePex package
\noautomath % you should always declare this after loading gb4e
%\usepackage{cgloss} %remove the extra space in the preamble and before the translation line

% The three commands below help number examples in fn. with roman i), ii), etc.
\makeatletter
\pretocmd{\@footnotetext}{
	\@noftnotefalse\setcounter{fnx}{0}%
	\renewcommand{\thexnumi}{\roman{xnumi}}
}{}{}
\apptocmd{\@footnotetext}{
	\@noftnotetrue
	\renewcommand{\thexnumi}{\arabic{xnumi}}
}{}{}


\usepackage[mcolblock]{leipzig}
\newleipzig{de}{de}{de1 的}
\newleipzig{sub}{sub}{subordinate operator}
\newleipzig{sfp}{sfp}{sentence final particle}
\newleipzig{exp}{exp}{experiential aspect}
\newleipzig{shi}{shi}{focus marker shi}
\newleipzig{ba}{ba}{ba}
\newleipzig{red}{red}{reduplication}
\newleipzig{tm}{tm}{topic marker}
%\usepackage{glossary-long}
%\setglossarystyle{long}

\makeglossaries %activate to make glossaries in the end


\usepackage{tipa} %ipa --> fontspec has problem with tipa
\usepackage{vowel}
\usepackage{tikz}
%\usepackage{tikz-qtree}


\usepackage{amsmath} %to enlarge the inventory of your symbols
\usepackage{amssymb}
\usepackage{MnSymbol} %for the meaning bracket

%%citation -----------------------

%\usepackage[backend=biber,style=apa,natbib=true]{biblatex}
%\DeclareLanguageMapping{british}{british-apa}
%\usepackage{csquotes}
%\addbibresource{}
\usepackage{apacite}
\AtBeginDocument{%
	\renewcommand{\BBAY}{{} }%% punctuation between authors and year
	\renewcommand{\BBN}{: }%% punctuation between year and page number
	\renewcommand{\BBYY}{, }%% punctuation between multiple years
}

% This is needed because the default authoryear style puts "In:"
% before a journal name, which is not standard in linguistics styles. So this code gets rid of that.
%\renewbibmacro{in:}{\ifentrytype{article}{}{\printtext{\bibstring{in}\intitlepunct}}}

%% notes ----------------
\usepackage[colorinlistoftodos]{todonotes}
\setuptodonotes{inline}
\newcommand{\todor}[1]{\todo[color=LightBlue1]{R: #1}}
\newcommand{\todod}[1]{\todo[color=Gold2]{D: #1}}
\newcommand{\todoi}[1]{\todo[color=MistyRose1]{I: #1}}
\newcommand{\todoq}[1]{\todo[color=OliveDrab3]{Q: #1}}
\newcommand{\todow}[1]{\todo[color=DeepPink1!65]{W: #1}}
\usepackage{marginnote}
